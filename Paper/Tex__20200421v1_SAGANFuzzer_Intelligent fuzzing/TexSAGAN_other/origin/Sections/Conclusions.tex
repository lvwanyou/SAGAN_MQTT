\section{Conclusions and Future works}
In this study, we propose an effective fuzzing methodology based on DCGAN to generate fake but plausible fuzzing data about ICPs. This methodology utilizes CNN to learn the spatial structure and distribution of real-world messages and generate similar data frames without knowing the detailed protocol specification. Allowing the convolution neural networks to learn message formats can save human effort and reduce time. In this manner, when testing other network protocols, we do not need to understand their specifications, which is convenient. We ultimately evaluate this method by fuzzing two safety-critical ICPs, including Modbus-TCP and EtherCAT. The results indicate that the proposed method has application potential to test a series of ICPs.

In future studies, we expect to create a more intelligent and more automated network protocol fuzzing system deployed to embedded devices. The system can apply the manner of online learning to learn protocol specifications or message formats of different protocols automatically. Considering the current situation, we intend to perform the study in the following aspects. First, we want to do a further exploration of other architectures to enhance our approach. Second, we will use our method to test other stateful ICPs, such as Profibus, Powerlink and future TSN. These protocols constitute an important part of most current ICPs. Finally, we intend to integrate each excellent architecture and processing module to form a hybrid model and a complete software system, which can deal with most network protocols, including stateful protocols and non-stateful protocols.